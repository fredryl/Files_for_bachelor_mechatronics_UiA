% This LaTeX was auto-generated from MATLAB code.
% To make changes, update the MATLAB code and export to LaTeX again.

\documentclass{article}

\usepackage[utf8]{inputenc}
\usepackage[T1]{fontenc}
\usepackage{lmodern}
\usepackage{graphicx}
\usepackage{color}
\usepackage{hyperref}
\usepackage{amsmath}
\usepackage{amsfonts}
\usepackage{epstopdf}
\usepackage[table]{xcolor}
\usepackage{matlab}

\sloppy
\epstopdfsetup{outdir=./}
\graphicspath{ {./Max_safe_forces_and_speed_images/} }

\begin{document}

\begin{par}
\begin{flushleft}
Here, maximum speed and power must be calculated on the body part with the lowest tolerance for injury/pain. Based on ISO 15066-2016
\end{flushleft}
\end{par}

\begin{matlabcode}
clc; close all; clear;
%  defenition of variable:
Beam_weigth = 0.39630912 % Kg, only beam. No ball.
\end{matlabcode}
\begin{matlaboutput}
Beam_weigth = 0.3963
\end{matlaboutput}
\begin{matlabcode}
Ball_weigth = 0.00283 % kg, Only the ball.
\end{matlabcode}
\begin{matlaboutput}
Ball_weigth = 0.0028
\end{matlaboutput}
\begin{matlabcode}
weigth_beam_with_ball = Beam_weigth+Ball_weigth % Kg, need to be corrected, but beam now without sensors 2.83g
\end{matlabcode}
\begin{matlaboutput}
weigth_beam_with_ball = 0.3991
\end{matlaboutput}
\begin{matlabcode}
beam_lengt_from_center =  0.4216 % m, Distance from center to the farest from sensor (bending moment).
\end{matlabcode}
\begin{matlaboutput}
beam_lengt_from_center = 0.4216
\end{matlaboutput}
\begin{matlabcode}

\end{matlabcode}


\matlabheading{Energy limit values based on the body region model.}

\begin{matlabcode}
max_juele_face = 0.11 % juele, Nm. page 28, tabel A.4. 
\end{matlabcode}
\begin{matlaboutput}
max_juele_face = 0.1100
\end{matlaboutput}


\matlabheading{Biomechanical limits}

\begin{matlabcode}
Maximum_permissible_force = 65 % N, Face masticatory muscle. Tabel A.2 page 24.
\end{matlabcode}
\begin{matlaboutput}
Maximum_permissible_force = 65
\end{matlaboutput}
\begin{matlabcode}

mass_Head = 4.4 % kg, page 27, tabel A.3.
\end{matlabcode}
\begin{matlaboutput}
mass_Head = 4.4000
\end{matlaboutput}
\begin{matlabcode}
m_ball = 0.00283 % kg
\end{matlabcode}
\begin{matlaboutput}
m_ball = 0.0028
\end{matlaboutput}
\begin{matlabcode}


% finding max velocity relativ to hiting the object.
m_R = (Beam_weigth/2+m_ball) % kg, formula A.4 page 29.
\end{matlabcode}
\begin{matlaboutput}
m_R = 0.2010
\end{matlaboutput}
\begin{matlabcode}
my = (1/mass_Head+1/m_R)^-1 % kg, formula A.3 page 29.
\end{matlabcode}
\begin{matlaboutput}
my = 0.1922
\end{matlaboutput}
\begin{matlabcode}


V_relativ_m = sqrt((max_juele_face*2)/my) % m/s max velocity of the end of beam tip. formula A.2 page 28.
\end{matlabcode}
\begin{matlaboutput}
V_relativ_m = 1.0699
\end{matlaboutput}
\begin{matlabcode}
V_relativ_mm = V_relativ_m*10^3 % mm/s
\end{matlabcode}
\begin{matlaboutput}
V_relativ_mm = 1.0699e+03
\end{matlaboutput}
\begin{matlabcode}
omega = V_relativ_m/(beam_lengt_from_center) % Rad/s 
\end{matlabcode}
\begin{matlaboutput}
omega = 2.5376
\end{matlaboutput}
\begin{matlabcode}


% finding the max safe torque of from servo
safe_t_off = Maximum_permissible_force*beam_lengt_from_center % Nm 
\end{matlabcode}
\begin{matlaboutput}
safe_t_off = 27.4040
\end{matlaboutput}
\begin{matlabcode}



\end{matlabcode}

\end{document}
